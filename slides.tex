% $Header$

\documentclass{beamer}


% Copyright 2004 by Till Tantau <tantau@users.sourceforge.net>.
%
% In principle, this file can be redistributed and/or modified under
% the terms of the GNU Public License, version 2.
%
% However, this file is supposed to be a template to be modified
% for your own needs. For this reason, if you use this file as a
% template and not specifically distribute it as part of a another
% package/program, I grant the extra permission to freely copy and
% modify this file as you see fit and even to delete this copyright
% notice. 



\mode<presentation>
{
  \usetheme{metropolis}

  \usecolortheme{seahorse}
  \usefonttheme[onlylarge]{structurebold}
  \setbeamerfont*{frametitle}{size=\normalsize,series=\bfseries}
  \setbeamertemplate{navigation symbols}{}

  \setbeamercovered{transparent}
  % oder auch nicht
}


\usepackage[english]{babel}
\usepackage{pygmentex}
\usepackage{pifont}
\usepackage{fdsymbol}

\setpygmented{lang=rust,font=\ttfamily\scriptsize}
% oder was auch immer

%\usepackage[latin1]{inputenc}
% oder was auch immer

%\usepackage{times}
%\usepackage[T1]{fontenc}
% Oder was auch immer. Zu beachten ist, das Font und Encoding passen
% m�ssen. Falls T1 nicht funktioniert, kann man versuchen, die Zeile
% mit fontenc zu l�schen.


\title[Die Programmiersprache Rust] % (optional, nur bei langen Titeln n�tig)
{Die Programmiersprache Rust}

\subtitle
{Sicher, nebenläufig... und schnell}

\author 
{Florian Gilcher}

\institute
{
  CEO und Rust-Trainer\\
  Ferrous Systems GmbH }

\date
{DeafIT 2018}

\subject{Informatik}

% TODO Ferrous-Logo
% \pgfdeclareimage[height=0.5cm]{university-logo}{university-logo-filename}
% \logo{\pgfuseimage{university-logo}}



\AtBeginSubsection[]
{
  \begin{frame}<beamer>{Gliederung}
    \tableofcontents[currentsection,currentsubsection]
  \end{frame}
}

%\beamerdefaultoverlayspecification{<+->}



\begin{document}

\newcommand{\one}{\ding{202}}
\newcommand{\two}{\ding{203}}
\newcommand{\three}{\ding{204}}

\begin{frame}
	\titlepage
\end{frame}


\begin{frame}{Whoami}

	\begin{itemize}
		\item
		      Rust-Programmierer und Trainer seit 2013
		\item
			  Projektmitglied seit 2015
		\item
			  Vorher 10 Jahre Rubyist
		\item
			  Programmierberater mit kleiner Firma
	\end{itemize}
\end{frame}

\begin{frame}{Gliederung}
	\tableofcontents

\end{frame}


\section{Die Sprache}

\subsection{Hintergrund}

\begin{frame}{Historie - Woher kommt Rust?}

	\begin{itemize}
		\item
		      Begonnen von Graydon Hoare in Jahr 2008.
		\item
			  Adoptiert von Mozilla Research ca. 2010.
		\item
		      Abgegeben von Graydon im Jahr 2013
		\item
		      Version 1.0 im Jahr 2015.
	\end{itemize}
\end{frame}

\begin{frame}{Was sind die Ziele von Rust?}

	\begin{itemize}
		\item Sicher:
		      \begin{itemize}
			      \item
			            Keine unsicheren Speicherzugriffe.
			      \item
			            Sichere Collections-API als default
		      \end{itemize}
		\item<2->
		      Nebenläufig
		      \begin{itemize}
			      \item
			            Compiler weiss über Nebenläufigkeit bescheid
			      \item
			            Verhinder unsichere Zugriffe über Grenzen hinweg
		      \end{itemize}
		\item<3->
		      Schnell:
		      \begin{itemize}
			      \item
			            Laufzeitgeschwindigkeit ähnlich von C
			      \item
			            Sicherheitschecks komplett zur Kompilierzeit
		      \end{itemize}
	\end{itemize}
\end{frame}


\begin{frame}{Was sind die Ziele von Rust?}

	\begin{itemize}
		\item Stabile, große Codebasen:
		      \begin{itemize}
			      \item
			            Kommuniziert viel Kontext lokal
			      \item
			            Manchmal etwas verbos
			      \item
			            Detailliertes Reporting von Fehlern
		      \end{itemize}
		\item<2-> Bei voller Kontrolle
		      \begin{itemize}
			      \item
			            Rust erlaubt Kontrolle über das Speicherlayout
			      \item
			            Unterscheidet zwischen rohen Daten und Referenzen
		      \end{itemize}
	\end{itemize}
\end{frame}

\begin{frame}{Was sind die Ziele von Rust?}
	Rust ist eine schlechte Sprache für Codebeispiele auf Folien.
\end{frame}

\begin{frame}{Wie sieht Rust aus?}
	\inputpygmented[lang=rust]{code/hello_world.rs}
\end{frame}

\begin{frame}{Wie sieht Rust aus?}
	\inputpygmented[lang=rust]{code/longer_example.rs}
\end{frame}

\begin{frame}{Wie sieht Rust aus?}
	\inputpygmented[lang=rust]{code/expanded_example.rs}
\end{frame}

\subsection{Basiskonzepte}

\begin{frame}{Basiskonzepte}
	Rust hat wenige Basiskonzepte. Diese sind aber fundamental und etwas ungewohnt.
\end{frame}

\begin{frame}{Mutabilität}
	Programmiersprachen sehen heute Daten oft als (semantisch) immutable oder mutabel an. Wo landet Rust?
\end{frame}

\begin{frame}{Mutabilität}
	\inputpygmented[lang=rust]{code/mutability_broken.rs}
\end{frame}

\begin{frame}{Mutabilität}
	\inputpygmented[lang=rust]{code/mutability_working.rs}

	\begin{itemize}
		\item Mutabilität ist eine Eigenschaft der Variable!
		\item Der Mutabilitätsmarker wird später weiter verwendet.
	\end{itemize}
\end{frame}

\begin{frame}{Mutabilität}
	Rust landet auf beiden Seiten - und verwendet diese Information später weiter!
\end{frame}

\begin{frame}{Besitz - Ownership}
	Alle Daten in Rust werden von genau einer Partei besessen.

	\begin{itemize}
		\item Daten im Besitz können beliebig geändert werden
		\item Es besteht garantiert exklusiver Zugriff!
		\item Der Besitzer muss die Daten aus dem Speicher entfernen
		\item Dies geschieht am Ende eines Scopes
	\end{itemize}
\end{frame}

\begin{frame}{Besitz - Ownership}
	\inputpygmented[lang=rust]{code/ownership.rs}
\end{frame}

\begin{frame}{Besitz - Ownership}
	Besitz kann abgegeben werden:

	\inputpygmented[lang=rust]{code/ownership_function.rs}
\end{frame}

\begin{frame}{Besitz - Ownership}
	
	\inputpygmented[lang=rust,escapeinside=||]{code/ownership_function_broken.rs}

	\ding{202} ist nicht erlaubt!\\

	Das verhindert effektiv unabsichtlicher Weiterverwendung von gelöschten Daten!
\end{frame}

\begin{frame}{Ausleihen - Borrowing}
	Das ist auf die Dauer etwas unpraktisch. Daher können Daten auf "verliehen" werden.
\end{frame}

\begin{frame}{Ausleihen - Borrowing}
	\inputpygmented[lang=rust]{code/borrowing.rs}

	Lingo: \& ist eine Referenz, Referenzen leihen in Rust aus
\end{frame}

\begin{frame}{Ausleihen - Borrowing}
	Einfach ausgeliehene Daten dürfen nicht verändert werden!
\end{frame}

\begin{frame}{Ausleihen - Borrowing}
	\inputpygmented[lang=rust]{code/mutable_borrowing.rs}
	Lingo: \&mut ist eine mutable Referenz.
\end{frame}

\begin{frame}{Ausleihen - Borrowing - Regeln}
	Dazu gibt es folgende Regeln:
	\begin{itemize}
		\item Normales (immutables) Ausleihen ist beliebig häufig möglich.
		\item Mutables Ausleihen ist nur exakt einmal möglich.
		\item Beide Regeln sind exklusiv.
	\end{itemize}
\end{frame}

\begin{frame}{Ausleihen - Borrowing - Regeln}
	Effektiv garantiert Rust damit, dass veränderbarer Speicher immer nur von einem Programmteil gesehen werden kann.
	Datenraces beim schreiben/lesen auf Speicherzellen sind damit nicht möglich.
\end{frame}

\begin{frame}{Beispiel}
	\inputpygmented[escapeinside=||,font=\ttfamily]{code/failing_borrow.rs}
	\ding{202} Initialisierung eines neuen Vektors\\
	\ding{203} Zeiger auf das dritte Element\\
	\ding{204} Drittes Element wird gelöscht


\end{frame}

\begin{frame}{Beispiel}
	\inputpygmented[lang=rust]{code/borrow_error.txt}

	Mutationsfehler sind auch in Programmen mit nur einem Thread möglich!
\end{frame}

\begin{frame}{Iteration}
	\inputpygmented[escapeinside=||]{code/iteration.rs}
\end{frame}

\begin{frame}{Iteration}
	ConcurrentModificationExceptions sind in Rust nicht möglich!
\end{frame}

\begin{frame}{Generics und Algebraische Datentypen}
	Rust erlaubt generische Programmierung. Generics in Rust sind ähnlich zu C++ Templates und Java Generics.

	\inputpygmented[escapeinside=||]{code/generics.rs}
	\ding{202} Collection-Typen sind generisch\\
	\ding{203} Typ-Alternative für positiven Fall\\
	\ding{204} Typ-Alternative für negativen fall
\end{frame}

\begin{frame}{Nebenläufigkeit}
	Rust fängt bestimmte Sorten Nebenläufigkeitsfehler zur Kompilierzeit ab.

	\inputpygmented[lang=rust]{code/threading_error.rs}
\end{frame}

\begin{frame}{Nebenläufigkeit}
	Rust erkennt das Bewegen von Daten über Nebenläufigkeitsgrenzen hinweg und prüft 2 zusätzliche Eigenschaften:

	\begin{itemize}
		\item Send: können die Daten den Kontext (z.B. Thread) wechseln?
		\item Sync: Sind die Daten synchronisisert?
	\end{itemize}

	Nur synchronisiserte Daten dürfen in Rust konkurrierend mutiert werden!
\end{frame}

\begin{frame}{Nebenläufigkeit}
	Praktischerweise beweist das auch, wenn Daten sicher geteilt werden können, weil sie garantiert nicht verändert werden.
\end{frame}

\begin{frame}{Beispiel}
	\inputpygmented[lang=rust]{code/two_functions.rs}

	Diese beiden Funktionen können nicht parallel auf denselben Daten laufen!
\end{frame}


\begin{frame}{Häufig}
   Das ist ein nahezu klassischer Speicherverletzungsfehler in fast allen Browsern.
\end{frame}

\begin{frame}{Nahe Zukunft: async/await}
  \inputpygmented[lang=rust]{code/async_await.rs}

  Rusts async/await ist laufzeitfrei.
\end{frame}

\begin{frame}{Tooling}
	Rust kommt:

	\begin{itemize}
		\item Mit einem modernen Paketmanager (Cargo)
		\item Einem Umgebungsmanager (rustup)
	\end{itemize}
\end{frame}

\begin{frame}{Goodies}
	Rust bietet darüber hinaus:

	\begin{itemize}
		\item Eine unsichere Subsprache für direkten Speicherzugriff
		\item Ein sehr gutes und direktes Foreign-Function-Interface basierend auf dem Platform-(C)-ABI
		\item Krosskompilierung zu momentan 60 targets, inkl. bare metal
		\item Sehr gutes Unterstützung für Kommandozeilen-Tools.
	\end{itemize}
\end{frame}


\begin{frame}{Ressourcenschonend}
	\begin{itemize}
		\item Die Sprache benötigt keinerlei Speicherallokation zum funktionieren
		\item "Pay what you use": Nicht angeforderte Funktionen landen garnicht erst in der Binary
		\item Optimierungsfreundlich: Standardtools zur Binary-Optimierung funktionieren
	\end{itemize}
\end{frame}

\begin{frame}{Integrierbarkeit}
	Rust möchte nicht unbedingt die Hauptsprache in einem Projekt sein, aber die beste Zweitsprache.
\end{frame}


\begin{frame}{Zusammenfassung}
	Rust ist eine C/C++-ähnliche Sprache, die:

	\begin{itemize}
		\item Speichersicher ist
		\item Keine(!) Laufzeitumgebung erfordert
		\item Features bietet, die auch Hochsprachen Konkurrenz machen
		\item Für sichere Parallelisierung ausgelegt ist
	\end{itemize}
\end{frame}

\begin{frame}{Überall}
	Rust funktioniert ohne Abstriche von Mikrocontrollern, über embedded Linux, das Smartphone, bis auf dem Server!
\end{frame}

\begin{frame}{Zugänglich}
	Rust ist nicht leicht, aber auch nicht schwer.

	Es kommt aber mit sehr guter Dokumentation und vielen Lernresourcen.
\end{frame}

\section{Das Projekt}

\begin{frame}{Rust als FOSS-Projekt}
	Rust ist:

	\begin{itemize}
		\item Inzwischen 3 Jahre alt
		\item überraschend weit adoptiert
		\item stark am wachsen
	\end{itemize}
\end{frame}

\begin{frame}{Adoption}

	\begin{itemize}
		\item Internet: Google Fuchsia, Dropbox, Mozilla Firefox, Mozilla Servo
		\item Spiele: Chucklefish, Ready at Dawn
		\item Embedded: mehrere IoT-Platformen, Robotikhersteller, Neuseeländer Feuerwehr
		\item Kultur: Schauspiel Dortmund
		\item Blockchain: gefühlt alle
	\end{itemize}

	Mehr als 100 Firmen, fast alle grossen Player (öffentlich)
\end{frame}

\begin{frame}{Projektgröße}
	\begin{itemize}
		\item 120 Teammitglieder
		\item über 2000 Contributor (Compiler und Kernsoftware)
		\item Kein zentralisiertes Management (Zeitzonen!)
		\item Explizit kein BDFL (Benevolent Dictator for Life)
	\end{itemize}
\end{frame}

\begin{frame}{Aktivität}
	Unter den 10 aktivsten Projekten auf GitHub
\end{frame}

\begin{frame}{Offenheit}
	\begin{itemize}
		\item Alle Entwicklung und Planung findet im Offenen statt
		\item Sprachentwicklung über RFCs (Request for Comments), jederzeit nachlesbar
		\item Starke Reviewkultur
	\end{itemize}

	Fast komplett textbasiert!
\end{frame}

\begin{frame}{Ansprechbar}
	\begin{itemize}
		\item Auf allen Repositories gibt es für Beginner markierte Issues (üblicherweise "E-Easy")
		\item Bei den meisten ist Mentoring angeboten (üblicherweise "E-Mentored")
		\item Wir helfen bei der Auswahl
	\end{itemize}

	Nicht wenige Leute fangen mit Compilerthemen an! Auch gerne mit schweren!
\end{frame}

\begin{frame}{Macht mit!}
	Wer Lust hat: sprecht mich an, ich suche mit euch ein Ticket!
\end{frame}


\begin{frame}{Macht mit!}
	Wer Lust hat: sprecht mich an, ich suche mit euch ein Ticket!
\end{frame}


\end{document}



